\documentclass{beamer}

\usepackage{xcolor}
\usepackage{stmaryrd}

\newcommand{\m}[1]{\texttt{#1}}
\newcommand{\e}[1]{\textbf{#1}}

\newcommand{\notb}[1]{not #1}
\newcommand{\andb}[2]{#1 \&\& #2}
\newcommand{\orb}[2]{#1 || #2}
\newcommand{\implb}[2]{#1 ==> #2}
\newcommand{\iffb}[2]{#1 <=> #2}

\newcommand{\true}{\m{true}}
\newcommand{\false}{\m{false}}
\newcommand{\mnotb}[1]{\m{not}\ #1}
\newcommand{\mandb}[2]{#1\ \m{\&\&}\ #2}
\newcommand{\morb}[2]{#1\ \m{||}\ #2}
\newcommand{\mimplb}[2]{\m{\implb{#1}{#2}}}
\newcommand{\miffb}[2]{\m{\iffb{#1}{#2}}}

\newcommand{\sem}[1]{\llbracket #1 \rrbracket}

\title{Dependent Types and Theorem Proving: \\Proving is programming in disguise}
\author{Wojciech Kołowski}
\date{May 2021}

\usetheme{Darmstadt}

\begin{document}

\section{Constructive logic: you already know it}

\begin{frame}{Constructive logic 1/2}
\begin{itemize}
	\item In constructive logic, \textbf{propositions ARE NOT either true or false}.
	\item In constructive logic we usually think about propositions \textbf{in terms of their proofs.}
	\item In everyday language and also in mathematics as it is usually practiced, a ``proof'' means an argument by which one human demonstrates the truth of a statement to another human.
	\item In constructive logic, a proof of $P$ is a certificate that $P$ holds, i.e. a formal object which \textbf{certifies that $P$ has been proven.},
	\item Meaning of propositions is determined by how we can prove them and how we can use them to prove other propositions.
\end{itemize}
\end{frame}

\begin{frame}{Constructive logic 2/2}
\begin{itemize}
	\item We shouldn't think about propositions as being either ``true'' or ``false'', but it's a deeply ingrained and hard to avoid way of thinking, so a translation:
	\item If we have a proof of $P$, we may think that $P$ is ``true''.
	\item If we have a proof of $\neg P$, we may think that $P$ is ``false''.
	\item If we have neither proof, we don't know anything about $P$.
\end{itemize}
\end{frame}

\begin{frame}{Propositions vs types}
\begin{itemize}
	\item There's a strange parallel going on between propositions and types.
	\item Types are, obviously, not either true or false -- they are inhabited by programs.
	\item A program $t$ of type $A$ is something that, after performing some computations, returns an element of type $A$.
	\item The meaning of a type $A$ is determined by how we can write programs of type $A$ and how we can use programs of type $A$ to write other programs.
\end{itemize}
\end{frame}

\begin{frame}{Propositions are types, proofs are programs}
\begin{itemize}
	\item This ``strange parallel'' is not a coincidence. There are no coincidences in mathematics!
	\item It is most often referred to as the Curry-Howard correspondence, after two out of many people who discovered it.
	\item But it is better presented as a set of slogans:
	\item \textbf{Propositions are types.}
	\item \textbf{Proofs are programs.}
	\item \textbf{Proving theorems is just writing programs.}
	\item ... and a few more, which we'll see shortly.
\end{itemize}
\end{frame}

\begin{frame}{True is the unit type 1/2}
\begin{itemize}
	\item There's the unit type \m{unit}.
	\item It's sole element is \m{()}.
	\item We can't do anything useful with it.
\end{itemize}
\end{frame}

\begin{frame}{True is the unit type 2/2}
\begin{itemize}
	\item There's the true proposition $\top$.
	\item It's sole proof is \m{()}.
	\item We can't conclude anything useful from it.
\end{itemize}
\end{frame}
	
\begin{frame}{Conjunction is the product type 1/2}
\begin{itemize}
	\item If \m{a} and \m{b} are types, then \m{a * b} is also a type.
	\item Elements of \m{a * b} are pairs \m{(x, y)}, where \m{x :\ a} and \m{y :\ b}.
	\item If we have a pair \m{x :\ a * b}, then \m{fst x :\ a} and \m{snd x :\ b}.
\end{itemize}
\end{frame}

\begin{frame}{Conjunction is the product type 2/2}
\begin{itemize}
	\item If $P$ and $Q$ are propositions, then $P \land Q$ is also a proposition.
	\item To prove $P \land Q$, we have to prove $P$ and we have to prove $Q$, so\dots
	\item \dots proofs of $P \land Q$ are of the form \m{(x, y)}, i.e. they are pairs where \m{x} is a proof of $P$ and \m{y} is a proof of $Q$.
	\item If $P \land Q$ holds, then we can conclude that $P$ holds and we can conclude that $Q$ holds, so\dots
	\item \dots if \m{x} is a proof of $P \land Q$, then \m{fst x} is a proof of $P$ and \m{snd x} is a proof of $Q$.
\end{itemize}
\end{frame}

\begin{frame}{Implication is the function type 1/2}
\begin{itemize}
	\item If \m{a} and \m{b} are types, then \m{a -> b} is also a type.
	\item Elements of \m{a -> b} are of the form \m{fun (x :\ a) -> e} -- they are functions which take an input \m{x} of type \m{a} and return \m{e} of type \m{b} as output.
	\item If we have a function \m{f :\ a -> b} and an \m{x :\ a}, then we we can apply \m{f} to \m{x}, written \m{f x}, to get an element of type \m{b}.
\end{itemize}
\end{frame}

\begin{frame}{Implication is the function type 2/2}
\begin{itemize}
	\item If $P$ and $Q$ are propositions, then $P \implies Q$ is also a proposition.
	\item To prove $P \implies Q$, we need to assume that $P$ holds and then provve $Q$ under this assumption, so...
	\item ... proofs of $P \implies Q$ are of the form \m{fun (p :\ P) -> q}, i.e. they are functions which take a proof of $P$ as input and return a proof of $Q$ as output.
	\item If $P \implies Q$ holds and $P$ holds, we can conclude that $Q$ holds, so...
	\item ... if \m{f} is a proof of $P \implies Q$ and \m{x} is a proof of $P$, then \m{f x} is a proof of $Q$.
\end{itemize}
\end{frame}

\begin{frame}{Disjunction is discriminated union}
\begin{itemize}
	\item If $P$ and $Q$ are propositions, then $P \lor Q$ is also a proposition.
	\item To prove $P \lor Q$, we need either to prove $P$ or to prove $Q$, so\dots
	\item \dots proofs of $P \lor Q$ are of the form \m{inl p}, where \m{p} is a proof of $P$, or of the form \m{inr q}, where \m{q} is a proof of $Q$.
	\item If $P \lor Q$ holds and $P \implies R$ holds and $Q \implies R$ holds, we can conclude that $R$ holds, so\dots
	\item \dots if \m{x} is a proof of $P \lor Q$, then we can match on \m{x} and retrieve the proofs of $P$/$Q$ and use them to prove $R$.
\end{itemize}
\end{frame}



\subsubsection{Falsity and negation}
% Introduce the empty type using the mechanism of
% algebraic data types.

% Introduce the proposition False as the
% "propositions are types" interpretation of the
% empty type.

% Introduce negation as an implication whose codomain
% is False. Explain the meaning of negation ~ P as
% there being some internal inconsistency in P.

% BEWARE! Remember to discuss ex falso carefully as this
% aspect of negation is neither intuitive to ordinary
% people nor to programmers (the empty type is not very
% useful in F# or Haskell).

% Optional: mention strong negation.

\section{Higher-order logic: you already know it}

\subsection{Predicates and relations}
% Explain the notions of predicate and relation and
% then introduce notation (i.e. A -> Type and A -> B -> Type).

\subsection{Universal quantifier is the dependent function type}
\subsection{Existential quantifier is the dependent pair type}
% Describe rules for quantifiers by referring back to dependent
% function and pair types from last lecture.
% Mention that F* can prove some propositions all by itself,
% using only the built-in SMT solvers.

% Moved:
% Make sure to strongly emphasize that in this
% interpretation logic is constructive and explain
% what that means.

\section{Induction is recursion}
% Start with a refresher on recursion.
% Examples: basic list functions -- append, reverse, etc.

% Then explain proofs by induction by analogy with recursive
% definitions.
% Examples: properties of basic list functions.

\section{Inductive predicates and relations}
% Explain how to interpret inductive families and predicates
% and relations. Examples: list permutations, being element
% of a list, there exists/all elements of a list satisfy
% some predicate, a list has duplicate elements, etc.

\subsection{Undecidability and generative thinking}
% Explain the difference between defining a function which
% checks if a property holds ("how to check" is a kind of
% top-down thinking) and defining a property as an inductive
% family ("how to generate all proofs of this" is a kind of
% bottom-up thinking).

\subsection{Proof relevance}
% Explain the notion of proof relevance, i.e. that there can
% be different proofs of some propositions, represented by
% different elements of the corresponding type. Example:
% being an element of a list.

\section{Equality}
% Some background on the classical definitions of equality
% by Aristotle and Leibniz.

\subsection{Definition and convertibility}
% Show how to define equality as an inductive family.

% Show some example proofs, like 2 + 2 = 4 or some
% more properties of list functions.

% Explain why this definition even makes sense by
% explaining how computation works and the notion
% of convertibility.

\subsection{Properties of equality}
% Prove basic properties of equality, like symmetry
% and transitivity.

% Also prove some characterizations of equality for
% some type formers, like pairs or sums.

\subsection{Caveat: equality of functions and types}
% Explain why we can't prove that extensionally equal
% functions are equal.

% Explain why we can't prove that isomorphic types are
% equal. Some silly example: list and list', where
% list' is just a primed copy of list.

\subsection{Caveats: decidable and heterogenous equality}
% Reminder: differences between differents kinds of
% "equality": decidable equality, homogenous equality
% and heterogenous equality.

\section{Axioms and classical logic}
% Explain how to declare axioms and that assuming an axiom
% breaks normalization (i.e programs built from axioms don't
% always compute).

% Example 1: functional extensionality axiom.
% Exmaple 2: excluded middle.

\section{How to find proofs}
% Because propositions are types and proofs are programs,
% finding proofs is basically the same as writing programs.
% All the techniques programmers already know, like splitting
% a function into subfunctions or refactoring some code into
% a separate functions have their exact equivalents (splitting
% a proof into lemmas and refactoring a part of a proofs into
% a lemma, respectively).

\section{Exercises}
% Some basic properties of connectives and quantifiers.
% Practice defining predicates and relations using inductive families.
% Implement some decision procedures for equality of numbers, prove
% some equational properties of numbers.

\begin{frame}{}
\begin{itemize}
	\item 
\end{itemize}
\end{frame}

\end{document}