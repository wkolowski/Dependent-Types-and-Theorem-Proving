\documentclass{beamer}

\usepackage{xcolor}
\usepackage{stmaryrd}

\newcommand{\m}[1]{\texttt{#1}}
\newcommand{\e}[1]{\textbf{#1}}

\newcommand{\notb}[1]{not #1}
\newcommand{\andb}[2]{#1 \&\& #2}
\newcommand{\orb}[2]{#1 || #2}
\newcommand{\implb}[2]{#1 ==> #2}
\newcommand{\iffb}[2]{#1 <=> #2}

\newcommand{\true}{\m{true}}
\newcommand{\false}{\m{false}}
\newcommand{\mnotb}[1]{\m{not}\ #1}
\newcommand{\mandb}[2]{#1\ \m{\&\&}\ #2}
\newcommand{\morb}[2]{#1\ \m{||}\ #2}
\newcommand{\mimplb}[2]{\m{\implb{#1}{#2}}}
\newcommand{\miffb}[2]{\m{\iffb{#1}{#2}}}

\newcommand{\sem}[1]{\llbracket #1 \rrbracket}

\title{Dependent Types and Theorem Proving: \\Proving is programming in disguise}
\author{Wojciech Kołowski}
\date{May 2021}

\usetheme{Darmstadt}

\begin{document}

\frame{\titlepage}

\begin{frame}{Plan of lectures}
\begin{itemize}
	\item Lecture 1: Programming with dependent types.
	\item \textbf{Lecture 2: Proving theorems with dependent types.}
	\item Lecture 3: Differences between programming and proving.
	\item Lecture 4: Examples of bigger programs and longer proofs.
	\item Lecture 5: A deeper dive into F*.
\end{itemize}
\end{frame}

\frame{\tableofcontents}

\section{Introduction}

\begin{frame}{Boolean ``logic''}
\begin{itemize}
	\item Being a programmer, you are good friends with the booleans, aren't you?
	\item There are two booleans, \m{true} and \m{false}.
	\item We can combine booleans \m{b} and \m{c} with the usual boolean functions:
	\item \mnotb{b} -- ``not \m{b}''
	\item \mandb{b}{c} -- ``\m{b} and \m{c}''
	\item \morb{b}{c} -- ``\m{b} or \m{c}''
	%\item We can also define less commonly used boolean functions:
	%\item \m{\implb{b}{c} = \orb{\notb{b}}{c}} -- implication, pronounced ``if \m{b} then \m{c}''
	%\item \m{\iffb{b}{c} = \andb{\implb{b}{c}}{\implb{c}{b}}} -- logical equivalence, pronounced ``\m{b} if and only if \m{c}''
\end{itemize}
\end{frame}

\begin{frame}{What is a logic}
\begin{itemize}
	\item Boolean logic is not an example of what logicians call a ``logic'', in the sense that it is not a ``logical system'', but merely a type with some unary and binary functions on it.
	\item A logic usually consists of:
	\item A definition of what \textbf{propositions} we're dealing with.
	\item A \textbf{semantics}, which tells us what these propositions mean.
	\item A \textbf{proof system}, which tells us which propositions can be proven and disproven.
	\item A \textbf{soundness theorem} which states that propositions proven true using the proof system are semantically true.
	\item Optionally, there may also be a \textbf{completeness theorem} which states that all semantically true propositions can be proven.
\end{itemize}
\end{frame}

\begin{frame}{Propositions}
\begin{itemize}
	\item A proposition asserts that something is the case, irrespectively of whether this really is the case or not.
	\item Math example: ``$4$ is a prime number.''
	\item Software example: ``For each input string $x$, if $x$ is not malformed, my program produces as output an array of length at most $10$.''
	\item Hardware example: ``This circuit implements addition of 16 bit integers.''
	\item Real world example: ``It's raining or I like trains.''
	\item \textbf{Beware! Formal logic is not very good for reasoning about the real world!}
\end{itemize}
\end{frame}

\begin{frame}{Propositional constants and connectives}
\begin{itemize}
	\item Propositions (usual letters: $P, Q, R$) are defined as follows:
	\item $\top$ -- the true proposition.
	\item $\bot$ -- the false proposition.
	\item $P, Q, R, \dots$ -- propositional variables.
	\item $\neg P$ -- negation, read ``not $P$''.
	\item $P \lor Q$ -- disjunction, read ``$P$ or $Q$''.
	\item $P \land Q$ -- conjunction, read ``$P$ and $Q$''.
	\item $P \implies Q$ -- implication, read ``$P$ implies $Q$'' or ``if $P$ then $Q$''.
	\item $P \iff Q$ -- logical equivalence, read ``$P$ if and only if $Q$''.
\end{itemize}
\end{frame}

\begin{frame}{Classical logic}
\begin{itemize}
	\item Classical logic is the most widely known/taught/used logical system in the world.
	\item In classical logic, \textbf{we think of propositions as being either true or false.}
	\item Therefore, classical logic is the logic in which truth values are the booleans.
	\item The truth value of a propositional variable is determined by a \textbf{valuation} $v : \text{Var} \to \text{Bool}$.
	\item If $v(P) = \true$, then $P$ is considered to be true.
	\item Otherwise it's considered false.
\end{itemize}
\end{frame}

\begin{frame}{Semantics of classical logic 1/2}
\begin{itemize}
	\item Given a valuation $v : \text{Var} \to \text{Bool}$, the truth value of a proposition can be determined with a recursive function $\sem{-} : \text{Prop} \to \text{Bool}$.
	\item $\sem{\top} = \true$
	\item $\sem{\bot} = \false$
	\item $\sem{P} = v(P)$, where $P$ is a variable.
	\item $\sem{\neg P} = \mnotb{\sem{P}}$
	\item $\sem{P \lor Q} = \morb{\sem{P}}{\sem{Q}}$
	\item $\sem{P \land Q} = \mandb{\sem{P}}{\sem{Q}}$
	\item $\sem{P \implies Q} = \morb{(\mnotb{\sem{P}})}{\sem{Q}}$
	\item $\sem{P \iff Q} = \sem{P}\ \m{==}\ \sem{Q}$
\end{itemize}
\end{frame}

\begin{frame}{Semantics of classical logic 2/2}
\begin{itemize}
	\item $P$ is satisfiable when $\sem{P} = \true$ for some valuation.
	\item $P$ is falsifiable when $\sem{P} = \false$ for some valuation.
	\item $P$ is a tautology when $\sem{P} = \true$ for all valuations.
\end{itemize}
\end{frame}

\begin{frame}{Example}
\begin{itemize}
	\item Example: the proposition $P \implies Q$ is satisfiable (for $v(P) = \true, v(Q) = \true$).
	\item It is also falsifiable (for $v(P) = \true, v(Q) = \false$).
	\item Therefore, it is not a tautology.
	\item Example: the proposition $P \land Q \implies Q \land P$ is a tautology.
	\item We have $\sem{P \land Q \implies Q \land P} = \morb{(\mnotb{(\mandb{v(P)}{v(Q)})})}{(\mandb{v(P)}{v(Q)})}$.
	\item For any values of $v(P)$ and $v(Q)$ we always get $\true$.
\end{itemize}
\end{frame}

\begin{frame}{The rest}
\begin{itemize}
	\item We can check whether a proposition is a tautology by trying all possible valuations, but there are exponentially many of them.
	\item We can do better by defining a proof system with some axioms and inference rules, which would allow us to \textbf{prove} that a proposition is a tautology without trying all valuations.
	\item We won't do that because \textbf{classical logic is not the right logical system for proving programs correct}.
	\item We will use constructive logic instead.
\end{itemize}
\end{frame}

\section{Constructive logic: you already know it}

\begin{frame}{Constructive logic 1/2}
\begin{itemize}
	\item In constructive logic, \textbf{propositions ARE NOT either true or false}.
	\item In constructive logic we usually think about propositions \textbf{in terms of their proofs.}
	\item In everyday language and also in mathematics as it is usually practiced, a ``proof'' means an argument by which one human demonstrates the truth of a statement to another human.
	\item In constructive logic, a proof is a formal object which \textbf{certifies that the given proposition has been proven}, in which case we say that the propositions holds.
	\item Meaning of propositions is determined by how we can prove them and how we can use their proofs to prove other propositions.
\end{itemize}
\end{frame}

\begin{frame}{Constructive logic}
\begin{itemize}
	\item If we have a proof of $P$, we may think of it as ``true'' (although we shouldn't think in terms of true and false).
	\item If we have a proof of $\neg P$, we may think that $P$ is ``false''.
	\item If we have neither proof, we don't know anything about $P$.
\end{itemize}
\end{frame}

\begin{frame}{Propositions are types, proofs are programs}
\begin{itemize}
	\item If $t$ is a proof of $A$, which we write as $t : A$, then we consider the proposition $A$ to be true.
	\item Otherwise, we don't know anything about $A$.
\end{itemize}
\end{frame}

% Explain the "propositions as types" paradigm.
% Make sure to strongly emphasize that in this
% interpretation logic is constructive and explain
% what that means.

\subsubsection{Function types are implications}
\subsubsection{Sum is disjunction}
\subsubsection{Product is conjunction}
\subsubsection{Unit is True}
% In the above subsubsections, explain the meaning of all connectives one by one.

\subsubsection{Falsity and negation}
% Introduce the empty type using the mechanism of
% algebraic data types.

% Introduce the proposition False as the
% "propositions are types" interpretation of the
% empty type.

% Introduce negation as an implication whose codomain
% is False. Explain the meaning of negation ~ P as
% there being some internal inconsistency in P.

% BEWARE! Remember to discuss ex falso carefully as this
% aspect of negation is neither intuitive to ordinary
% people nor to programmers (the empty type is not very
% useful in F# or Haskell).

% Optional: mention strong negation.

\section{Higher-order logic: you already know it}

\subsection{Predicates and relations}
% Explain the notions of predicate and relation and
% then introduce notation (i.e. A -> Type and A -> B -> Type).

\subsection{Universal quantifier is the dependent function type}
\subsection{Existential quantifier is the dependent pair type}
% Describe rules for quantifiers by referring back to dependent
% function and pair types from last lecture.
% Mention that F* can prove some propositions all by itself,
% using only the built-in SMT solvers.

\section{Induction is recursion}
% Start with a refresher on recursion.
% Examples: basic list functions -- append, reverse, etc.

% Then explain proofs by induction by analogy with recursive
% definitions.
% Examples: properties of basic list functions.

\section{Inductive predicates and relations}
% Explain how to interpret inductive families and predicates
% and relations. Examples: list permutations, being element
% of a list, there exists/all elements of a list satisfy
% some predicate, a list has duplicate elements, etc.

\subsection{Undecidability and generative thinking}
% Explain the difference between defining a function which
% checks if a property holds ("how to check" is a kind of
% top-down thinking) and defining a property as an inductive
% family ("how to generate all proofs of this" is a kind of
% bottom-up thinking).

\subsection{Proof relevance}
% Explain the notion of proof relevance, i.e. that there can
% be different proofs of some propositions, represented by
% different elements of the corresponding type. Example:
% being an element of a list.

\section{Equality}
% Some background on the classical definitions of equality
% by Aristotle and Leibniz.

\subsection{Definition and convertibility}
% Show how to define equality as an inductive family.

% Show some example proofs, like 2 + 2 = 4 or some
% more properties of list functions.

% Explain why this definition even makes sense by
% explaining how computation works and the notion
% of convertibility.

\subsection{Properties of equality}
% Prove basic properties of equality, like symmetry
% and transitivity.

% Also prove some characterizations of equality for
% some type formers, like pairs or sums.

\subsection{Caveat: equality of functions and types}
% Explain why we can't prove that extensionally equal
% functions are equal.

% Explain why we can't prove that isomorphic types are
% equal. Some silly example: list and list', where
% list' is just a primed copy of list.

\subsection{Caveats: decidable and heterogenous equality}
% Reminder: differences between differents kinds of
% "equality": decidable equality, homogenous equality
% and heterogenous equality.

\section{Axioms and classical logic}
% Explain how to declare axioms and that assuming an axiom
% breaks normalization (i.e programs built from axioms don't
% always compute).

% Example 1: functional extensionality axiom.
% Exmaple 2: excluded middle.

\section{How to find proofs}
% Because propositions are types and proofs are programs,
% finding proofs is basically the same as writing programs.
% All the techniques programmers already know, like splitting
% a function into subfunctions or refactoring some code into
% a separate functions have their exact equivalents (splitting
% a proof into lemmas and refactoring a part of a proofs into
% a lemma, respectively).

\section{Exercises}
% Some basic properties of connectives and quantifiers.
% Practice defining predicates and relations using inductive families.
% Implement some decision procedures for equality of numbers, prove
% some equational properties of numbers.

\begin{frame}{}
\begin{itemize}
	\item 
\end{itemize}
\end{frame}

\end{document}