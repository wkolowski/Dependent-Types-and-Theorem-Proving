\documentclass{beamer}

\title{Dependent Types and Theorem Proving: \\Introduction to Dependent Types}
\author{Wojciech Kołowski}
\date{March 2021}

\usetheme{Darmstadt}

\begin{document}

\frame{\titlepage}
\frame{\tableofcontents}

\section{Intro: greetings, prerequisites, learning outcomes, plan of lectures, F*}
% The lectures are mostly aimed at people who know some F# and are comfortable
% with pattern matching, recursion, algebraic data types etc.

% Learning outcomes:
% - after listening to the lectures people shouldn't be scared of dependent
%   types and formal proofs anymore
% - if additionally they do some exercises, they should end up pretty good at
%   basic proving (up to quantifiers), but maybe still struggling with stuff
%   like defining inductive predicates

\section{General idea}
% A definition of dependent types (types that can depend on values) and an
% example -- a good one would be the type of MxN matrices.

\section{First-class types}

\subsection{What does ``first-class'' mean?}
% Explain the notion of "first-class" using the (lack of) first class functions
% in C (and maybe something else) as an example.

% Then introduce the universe of types. At this stage we will pretend that
% Type : Type. Mentioning the universe hierarchy should be postponed to lecture 3.

\subsection{Computing with types}
% Show that our functions can now juggle types the same way they can juggle
% integers or strings.

% Leading example: compute printf's type from format string.

\section{Dependent functions}
% Introduce pi types by analogy with ordinary function type.
% Always call it "dependent function type" and use F* notation
% (i.e. (x : A) -> P x) to minimize confusion.

% Show some silly examples, like returning results of different types in
% different if-branches. Maybe compare this to dynamic typing and underline
% the differences.

\subsection{Example: typesafe printf}
% Finish the typesafe printf example.

\subsection{How to express polymorphism}
% Polymorphic functions are just dependent functions with Type arguments.
% Mention how this relates to type constructors, i.e. list : Type -> Type.

\subsection{Implicit arguments and how to read function signatures}
% Mention that with dependent types, we can infer some arguments from
% the types of others, so we don't want to explicitly write all arguments.

\section{Dependent pairs}
% Introduce sigma types by analogy with ordinary products.
% Always call it "dependent pair type" and use F* notation
% (i.e. (x : A) * P x) to minimize confusion.

\subsection{Example: everybody loves filling forms}
% An example, something along the lines of how to use dependent pairs
% to avoid forms with "not applicable" fields.

\section{Optional: some arithmetic}
% An optional section.

\subsection{Sigma}
% Introduce the sigma notation for dependent pair types.
% Explain it by parallels with math's sigma notation for
% addition. Introduce the name "dependent sum (type)".

\subsection{Pi}
% Introduce the pi notation for dependent function types.
% Explain it by parallels with math's pi notation for
% multiplication. Introduce the name "dependent product (type)".

\subsection{Plus, cross and arrow}
% Explain that dependent pair types are a generalization of
% both sum types and product types.

% Explain that dependent function types are a generalization of
% both product types and function types.

\section{Inductive families}
% Refresher on ordinary algebraic data types.
% Introduce inductive families by analogy with ordinary algebraic
% data types.

\subsection{Example: length-indexed lists}
% Vectors - the canonical example of an inductive family.
% Mention problems with partial functions like head and tail
% and how to turn them into total functions on vectors.

\subsection{Example: enforcing protocols}
% Look up Idris tutorials/manuals/books for some example
% on how to use inductive families for enforcing protocols,
% i.e. for using sockets.

\section{Intrinsic vs extrinsic style}
% Is it better to define indexed inductive types like vectors
% or go with ordinary inductive types and constrain them with
% predicates using dependent pairs? Well, neither and this
% leads us to...

\section{Refinement types}
% Introduce refinement types, using F* notation (i.e. x : int{x > 5}).
% Go over the head/tail example again, but this time using refinements
% on ordinary lists.

\section{Exercises}
% No idea yet.

\begin{frame}{}
\begin{itemize}
	\item 
\end{itemize}
\end{frame}

\end{document}