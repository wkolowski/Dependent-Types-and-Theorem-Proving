\documentclass{beamer}

\title{Dependent Types and Theorem Proving: \\Differences between programming and proving}
\author{Wojciech Kołowski}
\date{March 2021}

\usetheme{Darmstadt}

\begin{document}

\frame{\titlepage}
\frame{\tableofcontents}

\section{The Big Lie: only some programs are proofs}
% Explain that during the last lecture we may have been
% led into believing some lies, namely that ALL programs
% are proofs, which is not true in general and not true
% neither for F# nor Haskell or any other mainstream
% functional language.

\section{Universes revisited}
% Explain Russell's paradox and then expose lie #1 that we have been pretending
% is true up to now (i.e. that Type : Type). Mention the universe hierarchy
% only briefly as a technical device to avoid the paradox, don't go into details.

\section{Why infinite loops are evil}
% Start with a pseudocode example showing how to prove
% a contradiction using a nonterminating program.
% Use this to motivate termination checking.

\section{Termination checking and well-foundedness}
% Introduce the notion of well-founded relations. This may be done in a
% simplified way in order not to bother people with definitions. Basically
% in this view well-founded = argument somehow decreases. Then discuss the
% most important ways of creating well-founded relations, i.e. products and
% images.

% At last discuss how termination checking is done in F*.

\section{Positivity checking}
% Give an example of a silly inductive type (along the lines of
% Bad = Bad -> Bad) which allows us to implement recursive
% programs without using recursion. Show how to use it to derive
% a contradiction.

% Then introduce the strict positivity criterion, i.e. inductive
% type I can't have argument that have an occurrence of I to the
% left of an arrow.

\section{Why throwing exceptions is evil}
% Show how to prove a contradiction by throwing an exception.
% Use this to motivate a deeper investigation into what stuff
% can programs do (besides number crunching of course) to be
% considered valid proofs.

\section{Purity and referential transparency}
% Introduce (or reintroduce, depending on audience) the notions of purity
% and referential transparency in the way they are understood in Haskell
% folk wisdom.

% Silly examples: Edinburgh is the capital of Scotland, Mary Jane loves Peter Parker.
% Less silly examples: random() + random() != 2 * random()

\section{Effects in F*}
% Introduce the mechanism of effects as a way of staticly checking what side
% effects programs are allowed to perform.

% Explain how the effect system can be used to reconcile proofs and programs so
% that programs can perform effects if allowed to, but proofs can't.

% An interesting example: show a proof of a contradiction inside an effect
% which allows nontermination and tell people that it's ok and safe.

\section{Exercises}
% Some repetitive exercises for concept testing:
% Exercise 1: (why) is this function terminating?
% Exercise 2: is this inductive type strictly positive?

% Less repetitive exercises: no idea.

\begin{frame}{}
\begin{itemize}
	\item 
\end{itemize}
\end{frame}

\end{document}