\documentclass{beamer}

\title{Dependent Types and Theorem Proving: \\F* and .NET and other practical matters}
\author{}
\date{March 2021}

\usetheme{Darmstadt}

\begin{document}

\frame{\titlepage}
\frame{\tableofcontents}

% This lecture will be given by Mateusz. List of possible topics:

% - modules and interfaces – no nested modules, how are they extracted

% - building and extraction – how the integration with our .net codebase works

% - subtyping – no subtyping on inductive types (some examples)

% - no polymorphism on effects – need to define functions couple of times in some context (can be limited with subeffecting e.g. total functions can be called in bigger contexts when working with total types)

% SMT solver related stuff:

% - can prove simple properties automatically, e.g. base cases of induction and step often requires only calling the property recursively with smaller arguments

% - smt patterns to apply lemmas automatically

% - F* encodes certain properties / code to a format understandable by SMT and makes it look for counterexample in a negated property – it can’t give you nice errors in case it fails to prove something

% - F* normalization (evaluation during type-checking) can help SMT with recursive functions and inductive types

% - extraction and erasure

% - Obj.magic not available -> dependent types not available at runtime, but can be used in refinements as it will be erased during extraction anyway

% - one needs to be careful with interop with F# -> API is stripped of refinements and nothing is checking if we call functions correctly from F# anymore – F* code was verified assuming a correct input would be passed)

% - tactics (showing how F* represents syntax (and proofs?), how to do some code inspection/generation – mostly examples

% - quotation and antiquotation – part of the above

% .NET interop and known limitations:

% - one can write any implementation of a function directly in F#, and F* will only verify other code against the interface, one can cheat heavily and easily shoot oneself in the foot

% Proof engineering:

% - Do not prove every detail manually – leave the work for SMT solver so that you have less fixing when you change something

% - Use asserts to make sure important properties hold at the time where you expect them to be true

% - Sliding admit technique (assume certain parts of the proof for debugging or to avoid building proofs from the bottom up)

\begin{frame}{}
\begin{itemize}
	\item 
\end{itemize}
\end{frame}

\end{document}