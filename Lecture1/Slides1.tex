\documentclass{beamer}

\usepackage{xcolor}

\newcommand{\m}[1]{\texttt{#1}}
\newcommand{\e}[1]{\textbf{#1}}

\title{Dependent Types and Theorem Proving: \\Introduction to Dependent Types}
\author{Wojciech Kołowski}
\date{March 2021}

\usetheme{Darmstadt}

\begin{document}

\frame{\titlepage}
\frame{\tableofcontents}

\section{Greetings}

\begin{frame}{General info}
\begin{itemize}
	\item The lectures will be held weekly on Fridays.
	\item Don't worry if you miss a lecture -- the slides are pretty massive and the talks are going to be recorded.
	\item Each lecture ends with some exercises which will help you familiarize yourself with F* and better understand the ideas covered in the talk.
	\item But you don't need to do them if you don't want to.
	\item This talks repo: \url{https://github.com/wkolowski/Dependent-Types-and-Theorem-Proving}
\end{itemize}
\end{frame}

\begin{frame}{Plan of lectures}
\begin{itemize}
	\item Lecture 1: Programming with dependent types.
	\item Lecture 2: Proving theorems with dependent types.
	\item Lecture 3: Differences between programming and proving.
	\item Lecture 4: Examples of bigger programs and longer proofs.
	\item Lecture 5: A deeper dive into F*.
\end{itemize}
\end{frame}

\begin{frame}{Learning outcomes}
\begin{itemize}
	\item You won't be scared of all those obscure, scary and mysterious names and notations.
	\item You will get basic familiarity with the ideas behind dependent types.
	\item You will begin to see logic and mathematics in a very different light, much closer to your day job (at least if you are a programmer working in F\#).
	\item If you do the exercises, you will gain a basic proficiency in F*.
\end{itemize}
\end{frame}

\begin{frame}{Introducing F*}
\begin{itemize}
	\item F* (pronounced ``eff star'') is a general-purpose purely functional programming language.
	\item Member of the ML family, syntactically most similar to F\#.
	\item Aimed at program verification.
	\item Dependent types.
	\item Refinement types.
	\item Effect system.
	\item Not a .NET language.
	\item Neither compiled nor interpreted -- it's a proof assistant, i.e. just a typechecker.
	\item To run a program, it has to be extracted to some other language, like F\#, OCaml, C or WASM, and then compiled.
\end{itemize}
\end{frame}

\begin{frame}{Don't worry, be happy, ask questions}
\begin{center}
	\color{red}
	I KNOW YOU DIDN'T UNDERSTAND THE PREVIOUS SLIDE, BUT BY THE END OF THESE TALKS, YOU WILL!
\end{center}
\end{frame}

\begin{frame}{Useful F* links}
\begin{itemize}
	\item \textbf{You can run F* inside your browser} (and have a nice tutorial guide you): \url{http://www.fstar-lang.org/tutorial/}
	\item GitHub: \url{https://github.com/FStarLang/FStar}
	\item Homepage: \url{http://www.fstar-lang.org/}
	\item Download: \url{http://www.fstar-lang.org/\#download}
	\item Papers (not approachable for ordinary mortals): \url{http://www.fstar-lang.org/\#papers}
	\item Talks/presentations (more approachable): \url{http://www.fstar-lang.org/\#talks} (some of these are quite approachable if you're interested)  
\end{itemize}
\end{frame}

\begin{frame}{Prerequisites}
\begin{itemize}
	\item To understand what we will be talking about, you should have a working knowledge of F\# and the basic concepts of functional programming, namely:
	\item Functions as first-class citizens, including higher-order functions.
	\item Algebraic data types, including sum types and product types.
	\item Pattern matching and recursion.
	\item Even if you know these, you may be unfamiliar with the particular names -- for example, ``sum types'' is a name used in academia and Haskell, but in F\# they are better known as ``tagged unions''.
\end{itemize}
\end{frame}

\begin{frame}{Code snippet no 1 - basics of F*}
\begin{itemize}
	\item We will now see some code that shows how these things look in F*.
	\item See the file \texttt{Lecture1/Prerequisites.fst}.
\end{itemize}
\end{frame}

\section{Why}

\begin{frame}{Why should we care about dependent types? 1/3}
\begin{itemize}
	\item Programs written in dynamically typed languages perform a lot of runtime checks.
	\item Beyond a certain size \textbf{dynamically typed software is hard to extend, refactor and maintain because errors manifest very late} in the development process, i.e. at runtime.
	\item Statically typed languages make the situation better, because they move typechecking to compile time, which means a lot of errors get caught much sooner.
	\item \textbf{Static typing is good}.
\end{itemize}
\end{frame}

\begin{frame}{Why should we care about dependent types? 2/3}
\begin{itemize}
	\item But in simple functional languages like F\# there's still plenty of runtime checks -- division by zero, taking the head of empty list and a lot of user-defined checks which throw exceptions in case of failure.
	\item With dependent types, all runtime checks can be turned into static checks -- \textbf{all errors are type errors}.
	\item This results in more extensible, refactorable and maintainable software (and also better performance -- less stuff to do at runtime).\item We can not only get rid of runtime checks, dependent types can also replace most unit tests and property tests.
	\item \textbf{Dependent types bring static typing to its limits}.
\end{itemize}
\end{frame}

\begin{frame}{Why should we care about dependent types? 3/3}
\begin{itemize}
	\item And when I say all errors are typing errors, I really mean it -- with dependent types, we can express all properties, formulate all specifications and describe all mathematical objects.
	\item \textbf{Dependent types reveal a deep connection between functional programming and logic}.
	\item Despite their great power, dependent types are easy to understand and significantly simplify the language design.
	\item Have you ever heard about fancy Haskell stuff like multi-param typeclasses, GADTs, higher-rank types, higher-kinded types, existential types and so on?
	\item No? No problem -- \textbf{with dependent types, we get all of that (and much more) for free}.
\end{itemize}
\end{frame}

\section{Examples}

\begin{frame}{Matrix multiplication}
\begin{itemize}
	\item We can only multiply matrices whose dimensions match, i.e. we can multiply an $n \times m$ matrix by a $m \times k$ and get an $n \times k$ matrix as a result.
	\item How to model this in our favourite programming language without dependent types?
	\item The best we can do is to have \textbf{a type of matrices} \m{Matrix} and then matrix multiplication has type \m{matmult :\ Matrix -> Matrix -> Matrix}.
	\item What happens when we call it with matrices of the wrong dimensions?
	\item $\m{matmult} \begin{bmatrix}1 & 2\\3 & 4\end{bmatrix} \begin{bmatrix}1 & 2 & 3\\4 & 5 & 6\\7 & 8 & 9\end{bmatrix}$ \textbf{is well-typed, but will throw} an \m{IllegalArgumentException} or some other kind of runtime error, or maybe it will crash even less gracefully.
\end{itemize}
\end{frame}

\begin{frame}{Matrix multiplication with dependent types}
\begin{itemize}
	\item In a language with dependent types we can create \textbf{a type of $n \times m$ matrices} \m{Matrix n m} and give multiplication the type \m{matmult : (n :\ $\mathbb{N}$) ->  (m :\ $\mathbb{N}$) -> (k :\ $\mathbb{N}$) -> Matrix n m -> Matrix m k -> Matrix n k}
	\item Now \m{matmult} is a function which takes five arguments: the three matrix dimensions and the two matrices themselves.
	\item After giving it the dimensions of the first matrix from the previous slide, \m{matmult 2 2} has type \m{(k :\ $\mathbb{N}$) -> Matrix 2 2 -> Matrix 2 3 -> Matrix 2 3}.
	\item It is clear that \m{matmult 2 2 k} $\begin{bmatrix}1 & 2\\3 & 4\end{bmatrix} \begin{bmatrix}1 & 2 & 3\\4 & 5 & 6\\7 & 8 & 9\end{bmatrix}$ \textbf{is not well-typed} for any \m{k}, because the last argument is of type \m{Matrix 3 3}, but an argument of type \m{Matrix 2 k} was expected.
\end{itemize}
\end{frame}

\begin{frame}{A nice paper}
\begin{itemize}
	\item Dependent types can also be used to keep track of units of measure.
	\item This is possible in F\# too, but it's a built-in feature of the compiler, whereas the dependently typed solution is much more principled.
	\item It is also composable -- we can keep track of both matrix dimensions and units.
	\item There's a nice paper about this: \textbf{Type systems for programs respecting dimensions} available at \url{https://fredriknf.com/papers/dimensions2021.pdf}
\end{itemize}
\end{frame}

\begin{frame}{Array access}
\begin{itemize}
	\item When accessing the $i$-th element of an array, $i$ must be smaller than the length of the array.
	\item How to model this in our favourite programming language without dependent types?
	\item We have \m{Array A}, \textbf{the type of arrays of \m{A}s}, and we can access its elements with a function \m{get :\ Array A -> int -> A}.
	\item What happens, when $i$ is greater than the length of the array? Or, what happens when $i$ is negative?
	\item \m{get [| 'a'; 'b'; 'c' ] 5} is well-typed, but will throw an \m{IndexOutOfBoundsException} or result in a segmentation fault.
\end{itemize}
\end{frame}

\begin{frame}{Array access with dependent types 1/2}
\begin{itemize}
	\item In a language with dependent types we can have \m{Array A n}, \textbf{the type of arrays of \m{A}s whose length is \m{n}}.
	\item Then we have a few possibilities to model the type of \m{get}.
	\item \m{get :\ (n :\ $\mathbb{N}$) -> Array A n -> (i :\ $\mathbb{N}$) -> i < n -> A}.
	\item In this variant, the fourth argument of \m{get} is a proof that the index isn't out of bounds (we will cover proofs in the next lecture).
	\item We can't prove \m{5 < 3}, so we don't have any proof to feed into \m{get 3 [| 'a'; 'b'; 'c' |] 5 :\ 5 < 3 -> Char}.
\end{itemize}
\end{frame}

\begin{frame}{Array access with dependent types 2/2}
\begin{itemize}
	\item \m{get :\ (n :\ $\mathbb{N}$) -> Array A n -> (i :\ $\mathbb{N}$\{i < n \}) -> A}.
	\item In this variant we use refinement types (which we will cover later today) to automatically guarantee that \m{i} isn't out of bounds.
	\item \m{get 3 [| 'a'; 'b'; 'c' |] 5} is not well-typed, because the typechecker can't prove \m{5 < 3}, and thus \m{5} is not of type \m{$\mathbb{N}$\{5 < 3\}}.
\end{itemize}
\end{frame}

\section{The idea}

\begin{frame}{We're getting serious}
\begin{itemize}
	\item The above slides present nice fairy tales\dots
	\item \dots but how do dependent types actually work?
	\item And how to use them in F*?
	\item And what can ordinary programmers use them for besides number crunching with matrices and arrays?
\end{itemize}
\end{frame}

\begin{frame}{Values and types}
\begin{itemize}
	\item To understand dependent types, first we have to understand \textbf{dependency}.
	\item And to understand dependency, we need to be aware of the distinction between \textbf{values} and \textbf{types}.
	\item By \textbf{values}, we mean the bread-and-butter of programming: numbers, strings, arrays, lists, functions, etc.
	\item It should be pretty obvious to you that in most languages, \textbf{types} are not of the same status as numbers or functions.
\end{itemize}
\end{frame}

\begin{frame}{Dependencies}
\begin{itemize}
	\item Dependency is easy to understand. In fact, if you know basic F\#, then you already know most of it, because in F\#:
	\item \textbf{Values can depend on values}: we can think that the sum \m{n + m} is a number that depends on the numbers \m{n} and \m{m}. This dependency can be expressed as a function: \m{fun (n m :\ int) -> n + m}.
	\item \textbf{Values can depend on types}: for example, the identity function \m{fun (x :\ 'a) -> x} depends on the type \m{'a}.
	\item \textbf{Types can depend on types}: for example, the F\# type \m{Set<'a>} depends on the type \m{'a}.
\end{itemize}
\end{frame}

\begin{frame}{Naming the dependencies}
\begin{itemize}
	\item I bet you spotted the pattern in the previous slide, but it's a good idea to also have a name for the feature provided by each kind of dependency.
	\item Values can depend on values: (first-class) \textbf{functions}.
	\item Values can depend on types: \textbf{polymorphism} (i.e. ``generics'').
	\item Types can depend on types: \textbf{type operators}.
\end{itemize}
\end{frame}

\begin{frame}{Dependent types}
\begin{itemize}
	\item There's yet another kind of dependency, which is not present in F\#, but is present in F* and is the topic of this lecture.
	\item \textbf{Types can depend on values}: \textbf{dependent types}.
	\item But what are dependent types good for? You have been living your whole life without them, after all!
\end{itemize}
\end{frame}

\begin{frame}{The running summary 1}
\begin{itemize}
	\item \textbf{Dependent types are types that can depend on values}.
\end{itemize}
\end{frame}

\section{First-class types}

\begin{frame}{Juggling dependencies}
\begin{itemize}
	\item Given a functional language like F\#, how to enable types to depend on values?
	\item Of course we want to retain the other kinds of dependencies (values on values, values on types, types on types).
	\item It turns out it's best \textbf{throw away all kinds of dependencies besides the basic one} (values on values)\dots
	\item \dots and then \textbf{turn types into values}!
\end{itemize}
\end{frame}

\begin{frame}{Values and types}
\begin{itemize}
	\item So now values encompass both old, ordinary values (integers, tuples, functions, etc.) and new values (types).
	\item This way we get all four kinds of dependencies:
	\item Ordinary values can depend on ordinary values.
	\item Ordinary values can depend on type values.
	\item Type values can depend on type values.
	\item Type values can depend on ordinary values.
\end{itemize}
\end{frame}

\begin{frame}{First-class types}
\begin{itemize}
	\item How do we turn types into values?
	\item In programming languages' parlance, this process is called \textbf{making types first-class citizens of the language}.
	\item But what does ``first-class'' mean, anyway?
\end{itemize}
\end{frame}

\begin{frame}{What does ``first-class'' mean?}
\begin{itemize}
	\item In C, we can define functions that take an \m{int} and return an \m{int}.
	\item But we can't define a function that takes a function from \m{int}s to \m{int}s and returns an \m{int}.
	\item This means that functions from \m{int}s to \m{int}s are not treated the same as \m{ints}.
	\item As far as C is concerned, we can say that \textbf{integers are first-class, but functions are not first-class} (thus, they are ``second-class'').
	\item But C has function pointers, so you may be skeptical when I claim it doesn't have first-class functions.
\end{itemize}
\end{frame}
	
\begin{frame}{Some heuristics}
\begin{itemize}
	\item \textbf{The concept of ``first-class'' is neither precisely defined nor exact}. Rather, it's more of a functional programming folklore that obeys the ``I know it when I see it'' principle.
	\item However there are some heuristics that can help you.
	\item \textbf{Something is first-class when it can be}:
	\item bound/assigned to variables.
	\item stored in data structures.
	\item passed to functions as an argument.
	\item returned from functions.
	\item constructed at runtime.
	\item nameless, i.e. it can exist without giving it any name.
	\item So, which of these is criteria is not fullfilled by C's function pointers?
\end{itemize}
\end{frame}

\begin{frame}{A first-class quiz}
\begin{itemize}
	\item Let's have a little quiz to check if you get it.
	\item \textbf{Are the below language features first-class in F\# or not?}
	\item Functions?
	\item Recursive functions?
	\item Arrays?
	\item Modules?
	\item Records?
	\item Types?
\end{itemize}
\end{frame}

\begin{frame}{A type-based definition of ``first-class''}
\begin{itemize}
	\item Heuristics from previous slides are nice\dots
	\item but I prefer to think about first-class-ness in a different way, which is better from the functional programming point of view.
	\item For a given programming, \textbf{a concept $X$ is first-class if there is a type of all $X$s}, loosely speaking.
	\item This means that a language has first-class functions if for any two types $A$ and $B$ there is a type $A \to B$ of all functions from $A$ to $B$.
\end{itemize}
\end{frame}

\begin{frame}{A first-class quiz}
\begin{itemize}
	\item Let's have a little quiz to check if you get it.
	\item \textbf{Are the below language features first-class in F\# or not?}
	\item Functions?
	\item Recursive functions?
	\item Arrays?
	\item Modules?
	\item Records?
	\item Types?
\end{itemize}
\end{frame}

\begin{frame}{A first-class quiz answers}
\begin{itemize}
	\color{green} \item Functions? \color{black} For any \m{'a} and \m{'b} there's a type of functions \m{'a -> 'b}.
	\color{red} \item Recursive functions? \color{black} There's no separate type of recursive functions, even though there's a syntactic distinction between \m{let} and \m{let rec}!
	\color{green} \item Arrays? \color{black} For any type \m{'a} there's a type of arrays, namely \m{array<'a>}.
	\color{red} \item Modules? \color{black} Modules don't have types, they have signatures. But signatures are not types, so modules are not first-class.
	\color{yellow}\item Records? \color{black} This one is mixed depending on how you understand it. On the one hand, for any kind of record you can imagine, there's a corresponding type. But on the other hand, there is no type of all record types.
	\color{red} \item Types? \color{black} There are types in F\# and there are type variables \m{a'}, \m{b'}, \m{c'} etc., but we can't assign them any type!
\end{itemize}
\end{frame}

\begin{frame}{Computing with first-class types}
\begin{itemize}
	\item Previously we learned that ``$X$ is first-class'' means that there is a type of all $X$s.
	\item For types, this means that we need to have a \textbf{type of types}.
	\item And that's it -- we don't need anything else.
	\item Note: the phrase ``types of types'' sounds (and looks) bad, so we will call it \textbf{the universe of types}, or in short, just \textbf{the universe}.
	\item Because types are first-class in F*, we can assign them to variables, pass them to functions as arguments and return them from functions, and even compute types by recursion.
\end{itemize}
\end{frame}

\begin{frame}{Code snippet no 2 - first-class types in F*}
\begin{itemize}
	\item It might a bit difficult to wrap your head around the idea of first-class types, so let's see how it plays out in F*.
	\item The code snippet can be found in \m{Lecture1/FirstClassTypes.fst}
\end{itemize}
\end{frame}

\begin{frame}{The running summary 2}
\begin{itemize}
	\item Dependent types are types that can depend on values.
	\item \textbf{In dependently typed languages:}
	\item \textbf{There is a universe -- a type whose elements are themselves types}.
\end{itemize}
\end{frame}

\begin{frame}{So far so good}
\begin{itemize}
	\item So far so good, but we still don't know how dependent types work.
	\item We saw some in the examples, but they were left unexplained.
	\item We also saw some more in the last code snippet, but those were the crudest and most primitive dependent types in existence.
	\item \textbf{Now that we have learned about first-class types and the universe of types, we can learn dependent types proper}.
\end{itemize}
\end{frame}

\section{Functions}

\begin{frame}{Dependent types by analogy}
\begin{itemize}
	\item We will introduce dependent types by analogy.
	\item \textbf{Each of the various kinds of dependent types out there is just a generalization of an ordinary non-dependent type that is well-known to functional programmers}:
	\item Dependent function types are a generalization of function types.
	\item Dependent pair types are a generalization of products.
	\item Dependent record types are a generalization of records.
	\item Inductive types are a generalization of algebraic data types.
\end{itemize}
\end{frame}

\begin{frame}{Type families}
\begin{itemize}
	\item In the coming slides, we will often refer to \textbf{type families}.
	\item \textbf{A family of types indexed by type \m{a}} is just a function \m{a -> Type}.
	\item There can be many indices, like in \m{a -> b -> Type}.
	\item We have already seen examples in the last code snippet:
	\item \m{vec :\ Type -> nat -> Type} is a family of types whose members \m{vec a n} are lists of length \m{n} and elements of type \m{a}
	\item \m{matrix :\ Type -> nat -> nat -> Type} is a family of types whose members \m{matrix a n m} are $n \times m$ matrices with entries of type \m{a}.
\end{itemize}
\end{frame}

\begin{frame}{Non-dependent functions}
\begin{itemize}
	\item Recall how ordinary function types work in F\#.
	\item If \m{a :\ Type} is a type and \m{b :\ Type} is a type, then there is a type \m{a -> b :\ Type} of functions that take an element of \m{a} and return an element of \m{b}.
	\item We create functions of type \m{a -> b} by writing \m{fun (x :\ a) -> e} where \m{e} is an expression of type \m{b} in which \m{x} may occur.
	\item If we have a function \m{f :\ a -> b} and \m{x :\ a}, then we we can apply \m{f} to \m{x}, written \m{f x}, to get an element of type \m{b}.
\end{itemize}
\end{frame}

\begin{frame}{Dependent functions}
\begin{itemize}
	\item Now, watch the analogy unfold\dots
	\item If \m{a :\ Type} is a type and \textbf{\m{b :\ a -> Type} is a family of types}, then there is a type \textbf{\m{(x :\ a) -> b x} of dependent functions} which take an element of \m{a} \textbf{named \m{x}} and return an element of \m{b x}.
	\item We create functions of type \m{(x :\ a) -> b x} by writing \m{fun (x :\ a) -> e} where \m{e} is an expression of type \m{b x} in which \m{x} may occur.
	\item If we have a function \m{f :\ (x :\ a) -> b x} and \m{x :\ a}, then we can apply \m{f} to \m{x}, written \m{f x}, to get an element of type \m{b x}.
	\item Hint: it's probably easiest to pronounce \m{(x :\ a) -> b x} as ``for all \m{x} of type \m{a}, \m{b} of \m{x}''. Thus is revealed the connection to logic, which we will see in the next lecture.
\end{itemize}
\end{frame}

\begin{frame}{More dependent functions}
\begin{itemize}
	\item Of course, we can iterate the dependent function type to get a type of functions whose output type dependent on the value of many inputs.
	\item \m{(x :\ a) -> b x}
	\item \m{(x :\ a) -> ((y :\ b x) -> c x y)}
	\item Dependent function type associates to the right, just like ordinary function type, so we can drop the parentheses. We can also drop all but the last arrow.
	\item \m{(x :\ a) (y :\ b x) (z :\ c x y) -> d x y z}
	\item \m{(x :\ a) (y :\ b x) (z :\ c x y) (w : d x y z) -> e x y z w}
	\item etc.
\end{itemize}
\end{frame}

\begin{frame}{Code snippet no 3 - dependent functions in F*}
\begin{itemize}
	\item Let's see how to use dependent functions in F*.
	\item See the code snippet \m{Lecture1/DependentFunctions.fst}
\end{itemize}
\end{frame}

\begin{frame}{The running summary 3}
\begin{itemize}
	\item Dependent types are types that can depend on values.
	\item In dependently typed languages:
	\item There is a universe -- a type whose elements are themselves types.
	\item \textbf{There is a type of dependent functions which are just like ordinary functions, but their output TYPE can depend on the VALUE of their input}.
\end{itemize}
\end{frame}

\end{document}