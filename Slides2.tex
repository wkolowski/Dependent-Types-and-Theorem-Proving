\documentclass{beamer}

\title{Dependent Types and Theorem Proving: \\Proving is programming in disguise}
\author{Wojciech Kołowski}
\date{March 2021}

\usetheme{Darmstadt}

\begin{document}

\frame{\titlepage}
\frame{\tableofcontents}

\section{Constructive propositional logic: you already know it}

\subsection{Propositional logic}
% Describe the syntax of propositional logic.
% Explain the meaning of terms like "proposition" and
% the natural-language meaning of logical connectives.

\subsection{Propositions are types, proofs are programs}
% Explain the "propositions as types" paradigm.
% Make sure to strongly emphasize that in this
% interpretation logic is constructive and explain
% what constructivity means. Point out differences from
% classical (i.e. boolean) logic.

\subsubsection{Function types are implications}
\subsubsection{Sum is disjunction}
\subsubsection{Product is conjunction}
\subsubsection{Unit is True}
% In the above subsubsections, explain the
% meaning of all connectives one by one.

\subsubsection{Falsity and negation}
% Introduce the empty type using the mechanism of
% algebraic data types.

% Introduce the proposition False as the
% "propositions are types" interpretation of the
% empty type.

% Introduce negation as an implication whose codomain
% is False. Explain the meaning of negation ~ P as
% there being some internal inconsistency in P.

% Optional: mention strong negation.

\section{First-order logic: you already know it}

\subsection{Predicates and relations}
% Explain the notions of predicate and relation and
% then introduce notation (i.e. A -> Type and A -> B -> Type).

\subsection{Universal quantifier is the dependent function type}
\subsection{Existential quantifier is the dependent pair type}
% Describe rules for 

\section{Inductive predicates and relations}

\section{Proof relevance}

\section{Equality}
\subsection{How to define equality}
\subsection{Undecidability}
\subsection{Difference between propositions and booleans}

\section{Intrinsic vs extrinsic style}

\section{Axioms and classical logic}

\section{Exercises}

\begin{frame}{}
\begin{itemize}
	\item 
\end{itemize}
\end{frame}

\end{document}