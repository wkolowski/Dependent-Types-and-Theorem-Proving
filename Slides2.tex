\documentclass{beamer}

\title{Dependent Types and Theorem Proving: \\Proving is programming in disguise}
\author{Wojciech Kołowski}
\date{March 2021}

\usetheme{Darmstadt}

\begin{document}

\frame{\titlepage}
\frame{\tableofcontents}

\section{Constructive propositional logic: you already know it}

\subsection{Propositional logic}
% Describe the syntax of propositional logic.
% Explain the meaning of terms like "proposition" and
% the natural-language meaning of logical connectives.

\subsection{Propositions are types, proofs are programs}
% Explain the "propositions as types" paradigm.
% Make sure to strongly emphasize that in this
% interpretation logic is constructive and explain
% what constructivity means. Point out differences from
% classical (i.e. boolean) logic.

\subsubsection{Function types are implications}
\subsubsection{Sum is disjunction}
\subsubsection{Product is conjunction}
\subsubsection{Unit is True}
% In the above subsubsections, explain the meaning of all connectives one by one.

\subsubsection{Falsity and negation}
% Introduce the empty type using the mechanism of
% algebraic data types.

% Introduce the proposition False as the
% "propositions are types" interpretation of the
% empty type.

% Introduce negation as an implication whose codomain
% is False. Explain the meaning of negation ~ P as
% there being some internal inconsistency in P.

% Optional: mention strong negation.

\section{First-order logic: you already know it}

\subsection{Predicates and relations}
% Explain the notions of predicate and relation and
% then introduce notation (i.e. A -> Type and A -> B -> Type).

\subsection{Universal quantifier is the dependent function type}
\subsection{Existential quantifier is the dependent pair type}
% Describe rules for quantifiers by referring back to dependent
% function and pair types from last lecture.

\section{Inductive predicates and relations}
% Explain how to interpret inductive families and predicates
% and relations. Examples: list permutations, being element
% of a list, there exists/all elements of a list satisfy
% some predicate, a list has duplicate elements, etc.

\subsection{Undecidability and generative thinking}
% Explain the difference between defining a function which
% checks if a property holds ("how to check" is a kind of
% top-down thinking) and defining a property as an inductive
% family ("how to generate all proofs of this" is a kind of
% bottom-up thinking).

\subsection{Proof relevance}
% Explain the notion of proof relevance, i.e. that there can
% be different proofs of some propositions, represented by
% different elements of the corresponding type. Example:
% being an element of a list.

\subsection{Equality}
% Some background on the classical definitions of equality
% by Aristotle and Leibniz.

% Show how to define equality as an inductive family.
% Prove basic properties of equality.

% Reminder: differences between differents kinds of
% "equality": decidable equality, homogenous equality
% and heterogenous equality.

\section{Axioms and classical logic}
% How to declare axioms and use them to regain classical logic.

\section{How to find proofs}
% Because propositions are types and proofs are programs,
% finding proofs is basically the same as writing programs.
% All the techniques programmers already know, like splitting
% a function into subfunctions or refactoring some code into
% a separate functions have their exact equivalents (splitting
% a proof into lemmas and refactoring a part of a proofs into
% a lemma, respectively).

\section{Exercises}
% Some basic properties of connectives and quantifiers.
% Practice defining predicates and relations using inductive families.

\begin{frame}{}
\begin{itemize}
	\item 
\end{itemize}
\end{frame}

\end{document}